\documentclass[../dejiny-rodu-prusiku.tex]{subfiles}

\begin{document}

% str 66 @ 81
\section{Odnož Výrov I - větve Výrov}

Potomci Marie Šteflové z Výrova, narozené tam Prusíkové.

1815 - 1844

Páté dítě Vojtěcha Prusíka, výrovského rychtáře a jeho ženy Anny roz. Fenclové byla dcera Marie. Narodila se ve Výrově v čísle 18, ve stavení, kde se říkalo "u Boudů", dne 20. 5. 1815. Právě v tom roce byl poražen císař Napoleon u Waterloo. Při narození nebyly Marii asi její sudičky nakloněny. Život měla krátký. Dne 26. 11. 1833 vdala se ve věku 18 let za sedláka Josefa Štefla ve Výrově č. 6. Rodina Šteflů hospodařila ve Výrové již v prvním pololetí 18. století, a na tom gruntě ještě dnes je potomek s tímto jménem. Je to tedy jedna z nejstarších rodin usedlých ve Výrové. Josef Štefl manžel Marie, narodil se 24. 12. 1809. Bylo to v čísle 6, ačkoliv před přečíslováním mělo toto stavení ve Výrově Číslo 12. Z tohoto manželství narodily se dvě děti, které dospěly. Byl to syn Josef Štefl, nar. 3. 2. 1838 a dcera Marie nar. 22. 8. 1840. Josef Štefl stal se pak dědicem gruntu a zemřel ve Výrově 12. 4. 1891. Marie provdala se k Soukupům do Potvorova a tam zemřela 4. 6. 1918. Marii Šteflové, roz. Prusíkové narodil se v roce 1841 syn Vojtěch, který brzy zemřel a 7. 12. 1844 narodila se jí dcerka Barbora. Toto děcko také brzy zemřelo, ale bylo osudné matce, neboť ta následkem tohoto porodu, zemřela již 27. 12. 1844. Bylo jí pouze 29 let. Žila tedy na gruntě u Šteflů jenom 11 let. Její manžel Josef oženil se pak v dubnu 1845 podruhé a to s Marií Škubalovou z Hodyně. S ní měl pak ještě řadu dětí. Josef Štefl zemřel 21. 7. 1879 a přežil svou první ženu Marii roz. Prusíkovou o 43 let.

Prvním dítětem Marie Šteflové roz. Prusíkové byl syn Josef. Narodil se 3. 2. 1838 ve Výrově č. 6. Ve svých třiceti letech oženil se s Annou Vlkovou, jedinou dcerou selskou z Hrádecka. Ta se pak ukázala v hospodářství jako výborná hospodyně a v řízení celého statku jistě předčila svého muže, Josefa. Josef Štefl zemřel 12. 4. 1891 ve Výrově. Měl čtyři děti. Jeho syn František Štefl narodil se 2. 3. 1873. Snad by byl dostal on, jako prvorozený, statek po svém otci, ale vzhledem k tomu, že nevykazoval řád­nou pracovitost a iniciativu, grunt mu dán nebyl. František Štefl zůstal, snad vlivem tohoto roztrpčení, až do smrti svobodný a vypomáhal pak doma co bylo třeba. Zemřel 13. 3. 1947.

Druhým dítětem Josefa a Anny Šteflových, byla dcera Josefa. Narodila se 11. 3. 1875 ve Výrově a provdala se k Urbanům ve Výrově, kde se dříve říkalo "u Půtů". Urbanové přišli do Výrova z Kopidla. Josefa Urbanová, roz. Šteflová měla osm dětí, které dospěly. Nejstarší byl Václav. Narodil se 6. 11. 1896 a bydlí ve Výrové č. 26. Ze dvou manželství má čtyři děti. Nejstarší Václav Urban nar. 31. 10. 1914 je zaměstnancem ČSD a žije v Podbořanech, Strejcova ul. č. 187. Má dvě děti. Jaroslav Urban je nar. 17. 4. 1941 a Eva nar. 27. 2. 1944. Druhý syn Josef Urban narodil se 17. 9. 1921. Je zaměstnancem státního statku v Manětíně, kde
% str 67 @ 82
bydlí v č. 157. Má synka Jiřího Urbana, nar. 22. 3. 1962. Další je Jaroslav Urban, nar. 5. 12. 1924, je náměstkem ředitele RaJ (Restaurace a jídelny) v okrese Sokolov. Bydli v Karlových Varech-Tuhnicích, Krymská 19. Má dceru Alenu nar. 1. 2. 1953 a syna Václava nar. 31. 1. 1964.Čtvrtým dítětem Václava Urbana je dcera Jarmila, nar. 29. 1. 1927. Má za manžela člena JZD Vaška, a bydlí v Borku č. 4 u Kozojed. Mají tři děti. Jarmila nar. 3. 11. 1950, Miloslav nar. 21. 2. 1953 a Vladislav nar. 23. 9. 1957.

Druhým dítětem Josefy Šteflové, provdané Urbanové je dcera Božena. Narodila se 30. 11. 1898 a je provdaná Sládková v Kopidle č. 8. Je členem JZD. Má tři děti. Její syn Václav Sládek nar. 27. 9. 1921 adoptoval cizí dítě. Je to synek Václav. Dcera Miluše  nar. 23. 10. 1923 je provdaná Juhová v Kopidlí č. 23 a pracuje se svým mužem ještě jako soukromníci. Má dvě dcery. Miluška Juhová nar. 13. 2. 1947 a Jitka Juhová nar. 10. 9. 1950. Třetím dítětem Boženy Sládkové je dcera Božena. Narodila se 26. 8. 1926 provdala se za truhláře Kovaříka a žijí v Teplicích I, Čapkova 1854. Mají dvě děti. Jaroslav nar. 11. 8. 1950 a dcera Jana nar. 7. 7. 1958.

Dalším dítětem Josefy Urbanové je dcera Marie. Narodila se ve Výrově 20. 1. 1901. Později provdala se za výrovského rodáka Vojtěcha Vopata. Ten byl zaměstnancem drah a spolu pak žili dlouho v Ústí nad Lab. Nyní jako důchodci bydlí v Praze 6, Ruzyně, K mohyle 456. Mají dvě děti. Syn Josef Vopat nar. 12. 12. 1927 je konstruktérem a bydlí v Praze - Malešicích, Černokostelecká 2016. Má dvě děti. Květa nar. 12. 5. 1959 a Blanka nar 24. 5. 1965. Druhým dítětem Marie a Vojtěcha Vopatových je dcera Libuše. Je zdravotnickou dietní sestrou a má za manžela vedoucího lázeňského domu "Rudá hvězda" v Mar. Lázních, Štolbu. Libuše narodila se 17. 12. 1936 a má synka Marka nar. 14. 8. 1964.

Čtvrtým dítětem Josefy Urbanové je syn Josef. Narodil se ve Výrově 9. 3. 1903. Později zakoupil si zemědělskou usedlost v Kožlanech č. 325 a tam je dnes členem JZD. Josef Urban má čtyři děti. Nejstarší dcera je Marie, prov. Krocová v Bučku č. 1. Narodila se ve Výrově 23. 6. 1928. Krocovi měli v Bučku hostinec a dnes pracují v JZD. Mají tři děti. Nejstarší Eva nar. 24. 12. 1948 0 je provdaná Černá a bydlí v Praze 9 Čakovice, Sokolská 357. Pracuje v televizi. Druhá je Bohunka nar. 13. 12. 1950 a Miloš Kroc nar. 15. 1. 1954. Syn Josef Urban nar. 19. 4. 1930 je svobodný a pracuje v Kožlanech v JZD. Dcera Božena je provdaná Klausová, v Kožlanech. Narodila se 3. 11. 1931 rovněž ve Výrově jako všichni její sourozenci. Má dvě děti, Vladimíra nar. 24. 1. 1960 a dceru Jarušku nar. 18. 6. 1955. Čtvrté dítě Josefa Urbana je dcera Vlasta. Narodila se 13. 10. 1934 a dnes je provdaná Knotová v Bučku č. 19. Má čtyři děti, Pavel nar. 27. 7. 1954, Jindřiška nar. 1. 9. 1955, Drahuše 27. 7. 1959 a Ivana Knotová nar. 17. 5. 1966. Knotovi pracují v JZD. Josef Urban zemřel  9. 2.  1968.

Třetím synem Josefy Urbanové byl Vojtěch. Narodil se  9. 5. 1905, vyučil se zámečnictví a zemřel mladý, svobodný již 26. 6. 1927.

% str 68 @ 83
Dalším dítětem Josefy Urbanové, roz. Šteflové je dcera Anna. Narodila se 8. 10. 1908 a provdala se za hajného Antoše, který dlouho působil v Trojanech. Nyní žijí oba jako důchodci ve svém domku v Sedlci č. 20. Oba jejich synové stali se lesníky. Inž. Jaromír Antoš nar. 8. 11. 1930 bydlí v Manětíne č. 7. Je zaměstnán na lesní správě.  Má dceru Jaromíru Antošovou nar. 3. 1. 1959. Druhý syn, inž. Jaroslav Antoš, narodil se 20. 12. 1931. Je zaměstnán u lesní správy v Trhanově na Domažlicku. Bydlí v Trhanově č. 98. Má dva synky. Jaroslav nar. 6. 8. 1957 a Kamil nar. 20. 3. 1959. Jeho manželkou je dcera sedláka Růžičky z Výrova, jehož otec koupil v roce 1912 usedlost "u Boudů". Tam od roku 1803 působil rychtář Vojtěch Prusík.

Dalším dítětem Josefy Urbanové, byl syn Jaroslav. Narodil se 12. 1. 1911, byl zaměstnancem železnic a zemřel svobodný 10. 11. 1958.

Posledním dítětem Josefy Urbanové, byla dcera Jarmila. Narodila se 23. 1. 1913 ve Výrově. Provdala se za zaměst­nance lesní správy Antoše v Kopidle č. 47. Měla řadu let slabou nervovou soustavu a ta také způsobila její předčasnou smrt. Utopila se 15. 6. 1964 v rybníku v Kopidle. Jarmila Antošová měla čtyři děti. Nejstarší Jarmila nar. 24. 5. 1934 provd. Novotná bydlí dnes v Rovince, číslo 78 u Habartova v okrese Sokolov. Má dva syny. Josef Novotný nar. 26. 3. 1963 a Václav Novotný nar. 16. 3. 1965. Druhá je Zdeňka provdaná za dělníka Nového, naro­dila se 11. 4. 1936 a bydlí v Kopidle č. 47. Její synové jsou Jaroslav nar. 17. 10. 1957 a Miroslav nar. 4. 2. 1961. Třetí je syn Václav Antoš. Narodil se 29. 5. 1943 a pracuje v traktorové stanici v Kralovicích. Bydlí v Lomanech č. 18 u Plas. Má dceru Ivanu nar. 16. 5. 1964, a synka Václava Antoše nar. 21. 8. 1965. Poslední je Jana nar. 8. 5. 1945, provdaná Rýdlová. Má synka Václava nar. 6. 1. 1967. Bydlí v Kaznějově. Má též dcerku Janu, nar. 2. 10. 1968.

Třetím dítětem Josefa Štefla, sedláka ve Výrové č. 6 byl syn Václav. Narodil se 18. 9. 1878. Ten se brzy uká­zal, na rozdíl od svého bratra Františka, býti dobrým hospodářem. Proto se také stal dědicem gruntu. Pro své schopnosti byl také šest let starostou Výrova. Této funkce ujal se po Vojtěchu Prusíkovi, který byl starostou Výrova několik desetiletí. Václav Štefl nakazil se v létě 1918 tyfem, který tam tehdy řádil, a podlehl mu 27. 8. 1918. Nedočkal se tedy již prohlášení československé samostatnosti. Václav Štefl měl pět dětí, které dospěly. Nejstarší je Marie Čihová  nar. 5. 3. 1903 a bydlí Na Hadačce č. 55. Měla tři děti. Její syn Rudolf Čihák narodil se 28. 12. 1924 a zemřel jako dvanáctiletý chlapec 15. 12. 1936. Syn Josef Čihák, narodil se 6. 3. 1926, bydlí se svou matkou ve Výrově-Hadačce č. 55 a má dvě děti. Helena nar. 15. 12. 1951 a Josef nar. 22. 9. 1954. Třetí dítě je Marie, provd. Aksamitová ve Výrově č. 35. Narodila se 4. 1. 1940 a má dvě děti. Pavlu nar. 18. 3. 1963 a Luboše nar. 11. 6. 1966. Čihovi i Aksamitovi pracují v JZD.

% str 69 @ 84
Druhou dcerou sedláka Václava Štefla z Výrova č. 6 byla dcera Anežka, provdala se za majitele staroby­lého hostince Dyškánky, Hocha, Na Hadačce č. 12. Narodila se 3. 9. 1904. Má dvě děti. Její dcera Marie nar. 8. 12. 1930 bydli v Praze 9, Nad Kolčavkou 1438. Má za muže konstruktéra Glűckseliga, sama je úřednicí. Mají dvě děti. Dcera Iva je nar. 25. 4. 1956 a syn Pavel je narozen 21. 12. 1958. Druhé dítě Anežky Hochové je syn Alois. Narodil se 20. 10. 1934, bydlí rovněž na Dyškánce ve Výrově-Hadačce č. 12 a pracuje v JZD. Má dceru Jiřinu nar. 3. 11. 1957 a synka Jiřího nar. 9. 7. 1962. Rod Hochů je ve Výrově již dlouho usedlý. Hospodu Dyškánku vystavěl v XVIII. stol. plasský klášter a po jeho zrušení Náboženský fond prodal roku 1788 panský hostinec "Duškánku" Václavu Hochovi, Němci z  Minic. Za hostinec bylo zaplaceno 602 zlatých rýnských. Dostala se tedy Anežka Šteflová do rodu, který žije na témže místě ve Výrově již 180 let.

Dalším dítětem Václava Štefla byl syn Josef. Narodil se 17. 3. 1906 ve Výrově č. 6. Stal se dědicem gruntu po svém otci, byl to výborný zemědělec. Při hledání všech potomků Marie Šteflové roz. Prusíkové, napomohl on neobyčejně ke zdaru tohoto pátrání. Zemřel poměrně brzy, dne 1. 12. 1964. Měl tři děti. Dcera Marie, nar. 18. 7. 1932 provdala se za úředníka Skuhrovce do Rokycan, kde bydlí v č. 799. Má dceru Danu nar. 13. 4. 1958. Druhá dcera Josefa Štefla, Zdeňka je narozena 21. 4. 1934. Provdala se za ředitele hospodářské školy a bydlí v Mýtě č. 13 u Rokycan. Má dceru Zdeňku nar. 31. 10. 1954. Na rodném gruntě ve Výrově č. 6 žije dnes syn Josef Štefl. Je již pátou generací rodu Šteflů, jehož pramátí se stala Marie Prusíková z Výrova č. 18. Josef Štefl narodil se 7. 7. 1941 a jeho dcerka Jana, šestá generace v rodě, narodila se 3. 10. 1962. Josef Štefl pracuje v JZD.

Třetím dítětem Václava Štefla je syn Václav. Narodil se 25. 7. 1913 a byl zemědělcem v Kočíně č. 19 u Kralovic. Má tři děti. Václav nar. 29. 1. 1946, dcera Zdeňka nar. 4. 5. 1948 a Jitka nar. 25. 6. 1957. Václav zemřel 23. 4. 1968.

Poslední je Božena Šteflová, narozena 26. 5. 1915. Provdala se za rolníka Koču do Žebnice č. 9. Oba tam hospodaří stále soukromě a mají vzdor tomu výborné výsledky. Božena Kočová má dvě děti. Marie nar. 30.12. 1938 je provdaná Bulínová a bydlí v Plzni—Slovanech, Motýlí ulice č. 28. Má dvě dcery: Jana nar. 23. 6. 1962 a Hana Bulínová nar. 10. 8. 1964. Syn Václav Koča nar. 8. 12. 1941 pracuje ve svém hospodář­ství s rodiči. Má synka Václava nar. 6. 5. 1966.

Posledním dítětem Josefa Štefla byla dcera Marie. Narodila se 1. 1. 1881 ve Výrově č. 6, provdala se později za rolníka Slámu ve své rodné obci do čísla 14. Marie Slámová zemřela 29. 9. 1960. Od Slámů již dávno předtím, než se tam Marie přivdala, odešli dva členové do Ameriky. Marie Slámová měla pět dětí. Nejstarší byla Marie. Narodila se 15.  5. 1903, provdala se za rolníka Mancla do  Kralovic č. 134. Zemřela 18. 7. 1967. Měla dvě děti. Její syn Václav Mancl nar. 25. 9. 1924 bydlí v Kralovicích a pracuje v JZD. Dcera Marie nar. 4. 4. 1942 je provdaná Horáková
% str @ 85
v Zábřehu na Moravě. Má synka Ladislava nar. 13. 12. 1966.

Druhou dcerou Marie Slámové roz. Šteflové, je Božena. Narodila se 27. 6. 1905 a je provdaná za zaměstnance drah Polívku. Bydlí v Kralovicích č. 515. Má dvě děti. Její syn Jaroslav  Polívka nar. 31. 5. 1933 bydlí v Kralovicích č. 550. Má syna Jaroslava nar. 5. 2. 1958, a druhého Pavla nar. 24. 1. 1963. Dcera Věra je provdaná Bulínová a bydlí Na Hadačce, č. 50. Narodila se 19. 6. 1935 a má synka Vla­dimíra nar. 9. 11. 1959.

Třetím dítětem Marie Slámové je Anna. Narodila se ve Výrově 19. 10. 1907 a žije jako provdaná Králová na Hadačce, č. 13. Pracuje v JZD. Její dcera Anna provd. Urbanová v Trojanech č. 3 narodila se 12. 11. 1940. Má synka Romana Urbana nar. 17. 7. 1964. Druhá dcera Božena, provd. Horynová, Na Hadačce, narodila se 9. 11. 1945.

Čtvrtá dcera Marie Slámové je Blažena. Narodila se 25. 3. 1913 ve Výrově č. 14, bydlí dosud v této usedlosti a jejím manželem je rolník Slach, dnes člen JZD. Blažena Slachová má dva syny. Její syn František nar. 27. 6. 1937 je instruktorem zemědělského učiliště v Nečtinech. Bydlí tam v místě Hrad Nečtiny č. 4. Má dcerku Hanu Slachovou nar. 1. 7. 1962. Druhým synem Blaženy je Jaroslav Slach, nar. 4. 2. 1940. Má dvě děti. Jaroslav nar. 1. 11. 1963 a Dana Slachová, nar. 10. 1. 1965.

Posledním dítětem Marie Slámové roz. Šteflové, byl konečně syn, pokřtěný Vojtěch. Narodil se 19. 12. 1916 a přiženil se do Dolního Hradiště č. 7 u Kozojed. Má tři syny. První syn Vojtěch nar. 2. 9. 1944 bydlí doma a pracuje v JZD. Má dcerku Janu nar. 4. 11. 1966. Druhý syn, Jiří Sláma nar. 14. 11. 1945 pracuje v traktorové stanici v Kralovicích a tamtéž bydlí. Má syna Petra nar. 10. 8. 1967. Třetí je Václav Sláma nar. 9. 11. 1947.

Druhým a posledním dítětem Marie Šteflové, roz. Prusíkové ve Výrově byla dcera Marie. Narodila se 22. 8. 1840. Provdala se za rolníka Františka Soukupa v Potvorově. Měla tři děti, které dospěly. Byl to syn Josef Soukup nar. 11. 4. 1866 a on pak převzal usedlost po rodičích. Zemřel 13. 1. 1951. Druhá byla dcera Josefa, nar. 10. 8. 1872 provdaná Romová v Chrašťovicich. Tam zemřela 30. 1. 1936. Třetí byl syn František Soukup, nar. 12. 3. 1875 v Potvorově. Byl hostinským v Žilově. Zemřel 17. 12. 1957 u své dcery v Plzni. Marie Šteflová provdaná Soukupová v Potvorově, často vzpomínala na svůj výrovský domov a přesto, že její maminka Marie roz. Prusíková zemřela když jí bylo čtyři roky a tedy si ji nepamatovala, hlásila se vždy ráda ke všem členům rodu Prusík. Ať žili v Potvorově, anebo Bílově, Sedlci a blízkém okolí, navštěvovala tyto příbuzné a oni ji. Zemřela jako poloslepá 4. 6. 1918 v Potvorově.

Její první syn Josef narodil se 1. 4. 1866. Sedlačil řadu let na rodné usedlosti, ale později hospodářství chátralo a dnes nejsou již v Potvorové stopy p0 tomto jejich stavení. Josef Soukup zemřel v Potvorově 13. 1. 1951.

% str 71 @ 86
Josef Soukup z Potvorova měl čtyři děti, které dospěly. Nejstarší je dcera Marie Soukupová, nar. 22. 9. 1897. Neprovdala se a žije dnes v Horním Hradišti u Plas se svou sestrou Ludmilou Krejčovou.

Další byl syn Bohuslav Soukup. Narodil se 23. 5. 1900 v Potvorově a přiženil se do Trojan č. 1. Tam míval hostinec a při něm hospodářství. Má dvě dcery. Bohuslava nar. 2. 12. 1925 je provdaná Šmídlová a žije v Lomu u Mostu č. 458. Má tři děti. Dcera Bohuslava je narozená 29. 9. 1947, syn Jiří Šmídl nar. 23. 1. 1950 a dcera Jana nar. 10. 10. 1958. Druha dcera Bohuslava Soukupa z Trojan je Marie, nar. 4. 2. 1931. Je provdaná za Josefa Kroupu, který se tam přiženil z Bukoviny. Pracují v JZD. Mají také tři děti. Jaroslava Kroupová nar. 9. 12. 1953, Václav nar. 1. 11. 1956 a Jiří nar. 23. 8. 1963. Bohuslava Šmidlová je nyní provdaná Hynková v Kralovicích 52.

Třetí dítě Josefa Soukupa z Potvorova je Václav. Narodil se tam 28. 10. 1905 a přiženil se do Hrádecka u Královic. Tam bydlí v č. 3 a je členem JZD. Má tři děti. Nejstarší je Olga nar. 3. 2. 1932 a je provdaná Prýmasová v Kozlovicích č. 53 u Nepomuka. Má dceru Olgu nar. 7. 11. 1956. Syn Václav Soukup je narozen 14. 10. 1938 a bydlí v Hrádecku s rodiči. Druhou dcerou Václava Soukupa je Vlasta, narozená 14. 4. 1941. Provdala se za technika Spojů. Pekárka a bydlí v Praze, Kobylisích, Sedlecká 31/808. Mají dvě děti. Pavel je narozen 10. 5. 1963 a Ivana 30. 4. 1965.

Poslední byla dcera Ludmila, narozená 9. 2. 1908 v Potvorově. Provdala se za zemědělce Krejču do Horního Hradiště č. 18 u Plas. Její syn Jaroslav Krejča, nar. 16. 12. 1944 zemřel po tragické leukémii jako dvanáctiletý, 24. 8. 1956. Jedno malé děcko se Ludmile utopilo v rybníčku a tak jedinou její útěchou je dcera Jarmila, nar. 7. 1. 1947.

Druhým dítětem Marie Soukupové, roz. Šteflové byla dcera Josefa. Narodila se v Potvorově 10. 8. 1872 a byla pak provdaná Romová v Chrašťovicích. Tam měli pěknou usedlost. Zemřela 30. 1. 1936. Josefa Romová měla toliko jednu dce­ru. Byla to Marie narozená 2. 7. 1898 v Chrašťovicích, zůstala ve své rodné usedlosti a byla později provdaná Hlousová. Zemřela bezdětná 17. 7. 1952.

Posledním dítětem Marie Soukupové z Potvorova, která se narodila jako Šteflová ve Výrově, byl syn František. Narodil se 12. 3. 1875. Usadil se v Žilově, kde měl hostinec až do roku 1945. Tam přímým jeho sousedem byla usedlost Bayerlů, kde byla přivdána Eliška, roz. Kabátová z Nevřeně, také člen našeho rodu, o níž jsme již vyprávěli jako o potomku Tomáše Prusíka z Horní Břízy. Když František Soukup nemohl dále již provozovat svou živnost, jak to bylo znemožněno i dalším tisícům jiných v tehdejších po­revolučních dobách, odstěhoval se ke své dceři do Plzně a tam zemřel 17. 12. 1957. Byl třikráte ženat a z těchto všech manželství narodily se mu dvě děti a to z prvního a třetího. Prvním jeho dítětem byla dcera Marie, nar. 2. 2. 1903. Její matka pocházející z Nevřeně po porodu zemřela. Marie provdala se za Josefa Škoulu, technického úředníka Škodovky, a žila s ním v hezkém domku v Pzni - Škvrňanech.

% str 72 @ 87
Otec její František Soukup, žil u ní až do smrti. Ona nepřežila jej dlouho, zemřela na rakovinu 30. 8. 1963. Marie Škoulová měla jediného syna Stanislava. Narodil se 27. 5. 1929, je technickým úředníkem Škodovky a bydlí v Plzni - Slovanech, Náměstí 16. Svůj domek ve Škvrňanech prodali. Stanislav Škoula má synka Pavla, nar. 10. 8. 1963.

Druhým dítětem Františka Soukupa je ze třetího manžel­ství syn František. Narodil se 7. 4. 1923 v Žilově, a byl tedy o 20 let mladší než jeho sestra Marie. Je inženýrem ve Škodovce a pro své odborné schopnosti je často vysílán do ciziny, i jako instruktor našich montérů. Zvláště dlouho působil v Egyptě. Bydlí v Plzni, Čechova 33. Inženýr František Soukup má dvě děti. Synka Františka nar. 4. 8. 1946 a Svatopluka nar. 28. 7. 949.

S rodinou Šteflů býval vždy úzký vztah. Asi také proto, že pramáti tohoto rodu Marie Prusíková se přivdala do usedlosti ve své rodné obci. S rodinou Prusíků nestýkaly se jen její vlastní děti a pak zase jejich potomci, ale také její nevlastní děti, které pocházely již z druhého manželství Josefa Štefla, když ona tak předčasně, mladá zemřela. Zvláště syn, František Štefl narozený z druhého manželství Josefa Štefla, měl k rodu Prusíků ten nejpřátelštější vztah. Když mu zemřel jeho otec v roce 1879 bylo mu patnáct let. Jeho poručníkem se stal Václav Prusík, bratr Marie Šteflové. Ten pak studoval s podporou Blažeje Prusíka, profesora v Klatovech, který byl o 2o let starší. Velkým jeho přítelem již od mládí až do konce života byl další bratr Blažejův, Josef Prusík. Studovali spolu v Klatovech a později se stýkali i v Praze, kde oba bydleli. František Štefl napsal krátké paměti své usedlostí ve Výrové a svého rodu, které osvět­lují velmi zajímavě život v minulém století v této vesnici. Ve svých vzpomínkách zmiňuje se také o krátké historii školy ve Výrově. Píše: "Na počátku roku 1869 přistěhoval se do Výrova učitel Fišer. Přezdívali mu „šlapant“. Byl už před tím v několika obcích, často se opíjel, stal se tam nemožným a musel zase "šlapati" do jiné obce. Obec najala u dolejšího Vopata větší světnici, učitel měl několik školních lavic, černou tabuli, na velkém archu psanou velkými ciframi násobilku, zavěšenou na stěně. Pak tam byl také černý kříž a školní síň zdobila také jeho postel. Jeho učení záleželo hlavně ve štědrém vyplácení březovou metlou přes ruce, jestliže žák mu dosti rychle neodpovídal. Násobilka se učila denně, společně a nahlas. Psaní se učilo nejdříve švabachem podle tehdejších úředních příkazů a teprve později latinkou. Učitel dostával "sobotáles“ za žáka. Na obědy chodil do stavení, kde měli školní děti. Na jedno dítě přišel jeden oběd, až zase přišla řada v celé vsi vystřídáním. Na sv. Řehoře v březnu, naučil každého žáka malý veršík o jarních polních pracích a vodil je od statku ke statku, aby všude své veršíky odříkali.

% str 73 @ 88
Někde dostal učitel Fišer od sedláka čtyrák a od selky nějaké vajíčko, nebo kousek másla, buchtu nebo koláč, což mu sjednaná stará žena odnášela do školy. To byly prastaré zvyky staré nepovinné školy a všelijakých učitelů. Po roce 1870 to novými školskými zákony přestalo. První a poslední učitel ve Výrově, Fišer vystěhoval se pak se svojí školou do Petrovic. Škola ve Výrově zanikla, takže existovala tam pouze dva roky. V roce 1871 na základě nových školských zákonů bylo úředně jednáno též o tom, aby ve Výrově byla zřízena a postavena škola, alespoň jednotřídní se dvěma odděleními, jak tehdy bylo zvykem. Aspoň pro ty nejmenší děti. Pro silný odpor dvou sedláků, starého Háži, který měl už tři dospělé syny a dolejšího Jirky Kozla, který už měl také dce­ru škole odrostlou, škola ve Výrově zřízena již nebyla. Často lidé říkávali, že výrovské malé děti vždy budou musit láteřit na ty dva sedláky až v mrazu, chumelenici nebo lijáku povlekou se do vzdálených Kralovic do školy."

František Štefl dále vzpomíná na polní hospodaření ve svém rodišti, Výrově: „Až do mých dětských let /kolem r. 1864-1867/ třetina všech výrovských polí ročně zahálela. Ouhořila, ležela ladem, aby nabrala prý nové síly k vydání úrody v příštím roce. Do těch dob se také všechno obilí u nás žalo srpy. Teprve v té době začal strýc Václav Prusík, který měl staršího bratra vikáře v Praze, u sv. Víta, na jeho upozornění zaváděti nové způsoby v rolnictví. Byl proti úhoření polí, začal je vápnit a za­vedl sekání pšenice a žita, kosou. K novotám, jež zavá­děl strýc Prusík, mnozí sousedé nejen u nás, ale i v širokém okolí vážné kroutili hlavou. Druhým rokem se již připojili k této novotě a třetím rokem sekali obilí kosami u nás již všichni i v širším okolí. Toho využil i můj otec Josef Štefl a začal vyráběti doma nejen hrábě jako dříve, ale i hrabišťata tj. násady pro kosy s malý­mi i velikými roubíky. Výrobky jeho šly velmi dobře na odbyt, dodával je dobrým i neznámým lidem, ale nepamatuji se, že bychom z této nové živnosti zbohatli.“

“V roce 1870,“ píše ve svých pamětech František Štefl, „byly v celém Rakousku a tudíž i v Čechách, zavedeny nové metrické míry a váhy. Přestaly lokte a sáhy, žejdlíky a věrtele, aspoň úředně. Je samozřejmé, že dlouho trvalo než se metry, kilometry, kilogramy a centy, hektolitry vžily. Na začátku žní v roce 1871 byl jsem na p0li se svou sestrou. Odpoledne jsme najednou spatřil1 nad Výrovem veliký sloup dýmu. Tehdy vyhořel na vršku Výrova Chalupák, hořejší Vopat a Vojtík. -Kráva tehdy stála 60 až 80 zlatých. Kůň až 200 zlatých. Holba piva /o něco víc než půl litru/ byla za čtyrák. /4 krejcary = 8 hal/. Pekař prodával housky za jeden nebo dva krejcary, podle velikosti. Cukr a sůl kupovali jsme domů na homole. Náš statek u Šteflů byl kdysi půllán. Celý lán se zachoval jen u Boudů /Prusíků/. Rozdělení našeho gruntu se však stalo dávno již v XVIII. století. Zdá se však, že tam kde se říká "u Šteflíků“ byla druhá, nová polovice rozděleného gruntu. A to nejen podle jména, ale i podle způsobu a zachovalosti stavby obou usedlostí.

% str 74 @ 89
František Štefl dále vzpomíná: „Jako kluk běhával jsem se džbánem pro pitnou vodu, když se naše studna po velkém dešti zakalila, za potok ke Slachům, k Hážům nebo do­lejším Maršům. Břidlicový útvar, který se prý táhne od Bílé hory od Prahy přes Kladensko a Rakovnicko, končí právě ve vršku nad naším statkem. Nedává dobrou pitnou vo­du a druhá strana Výrova za potokem měla vždy lepší pitnou vodu. Pamatuji se, že v roce 1872 nebo 1873 rozšiřovala se všude pověst, že jednoho dne v pravé poledne, veliká kometa narazí na naši zemi a celou ji zničí. Čekal jsem toho dne u potoka, aby mne kometa nespálila, ale nic se nestalo, jen ve Vídni v ten čas nastal na burze velký úpadek a zasáhl mnohé i na venkově zakládané "ouly", to je kupecké obchody, zvláště v Potvorově, Kralovicích a jinde. - Jediná vetší zahrada ve vsi byla u Boudů, kde hospodařil můj někdejší poručník, Václav Prusík. Pamatu­ji se také živě na den 25. května 1872. Šli jsme jako děti, ráno do školy do Kralovic. Před samým polednem zatáhla se obloha černými mraky, blýskalo se a hrozné hřmelo a brzo spustil se liják. Všude se valilo strašně vody, lou­ky pod farou a pěšinka jimi jdoucí, byly zatopeny vyso­ko přívalem vod. Teprve před samým večerem, když déšť již přestal, přijela pro nás z Dyškánky Hochovic bryčka a odvezli nás do Výrova. Celý náš dvůr byl zaplaven, zeď zahrádky do návsi byla stržena a jinde ve Výrově také mnoho škody. I soudnímu sluhovi Šimlovi v Kralovicích od­nesla voda malý baráček na lukách a byl mu postaven později nový pěkný, zděný domek u silnice na kraji města. Teprve třetí den jsme se dozvěděli jakou zkázu na majetku a na životech způsobila tato průtrž a potom povodeň na řece Střelce v Plasích, Nebřežinech, a jinde.

Tím končíme vyprávění o osudu a životě Marie Prusíkové, provdané Šteflové ve Výrově a připojili jsme k tomu několik vzpomínek na mladá léta jejího nevlastního syna Františka Štefla. Přestože Marie Šteflová narozená u Boudů ve Výrově, zemřela tak mladá a měla jen dvě děti,které dospěly, zanechala po sobě až k dnešku dosti velký počet potomků. Je to 157 lidí s krví prusíkovskou a z toho již 20 mrtvých.
\end{document}
