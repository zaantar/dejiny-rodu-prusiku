\documentclass[../dejiny-rodu-prusiku.tex]{subfiles}

\begin{document}

% str 37 @ 49
\section{Odnož Horní Bříza II - větve Výrov}

Potomci Rosalie Prusíkové z Výrova provdané Karezové.

1807 - 1851

Druhé dítě Vojtěcha Prusíka a Anny rozené Fenclové z Výrova byla Rosalie. Byla jen o dva roky mladší než její sestra Anna. Narodila se 3. 9. 1807. Od roku 1822 byla Anna provdaná již v Horní Bříze jako Štěpánková a brzy potom dohodl se sňatek Rosalie také do této obce. Dne 6. června 1826 vdávala se z Výrova devate­náctiletá Rosalie za sedláka Vojtěcha Kareza. Karezo­vé byli v Horní Bříze již dlouho předtím usedlí. Na křestním listě Vojtěcha Kareza je psáno jeho příjmení s "s" a mnozí jeho potomci píší se tak i dnes. Vojtěch Karez měl grunt č. 3 na staré návsi. Narodil se 22. 4. 1805. Zemřel 28. 9. 1864. Jeho žena Rosalie byla již třináct let po smrti. Zemřela náhle na mrtvici. Bylo to v době rozkvetlé přírody dne 9. 5. 1851. Jejich grunt č. 3 byl pak rozdělen 14. 12. 1850 mezi syny Josefa a Antonína. V čís­le 3 hospodařil Josef a v nové usedlosti, která dosta­la číslo 71, jeho bratr Antonín Karez. Rodina Vojtěcha a Rosalie Karezových zůstávala vždy v nejpřátelštějším styku s Výrovem. Když bratr Rosalie, Václav Prusík osi­řel po otci v roce 1841, stal se jeho poručníkem na dobu téměř dvou let jeho švagr, Vojtěch Karez. Dnes v obou usedlostech žijí sice ještě příbuzní, ale ne již se jménem Karez.

Rosalie s Vojtěchem měla čtyři děti, které dospěly. Nejstarší byl syn Antonín, který se narodil 29. 1. 1830 a zemřel v Horní Bříze v č. 71, dne 31. 7. 1908. Druhý byl Josef nar. 22. 1. 1833, hospodařil v čísle 3 od roku 1852 a tam zemřel 29. 12. 1904. Třetí syn Jan narodil se 6. 8. 1838. Na toho již doma nic nezbylo a přiženil se proto do Nýnic u Plzně. Zemřel pak u své dcery v Plzni dne 17. 5. 1918. Čtvrtým dítětem byla Marie, narodila se 12. 2. 1843, provdala se za Vojtěcha Maška, sedláka v Třemošné a tam zemřela 16. 6. 1924.

Antonín Karez, prvorozený syn narodil se 29. 1. 1830. Hospodařil V Horní Bříze č. 71. Měl tři syny: První byl Josef nar. 11. 10. 1852. Žil v Ledcích, Horní Bříze a dlou­ho v Plzni, kde zemřel 17. 7. 1938. Druhý byl Václav Karez nar. 8. 8. 1859, hospodařil v č. 71 a zemřel tam 21. 1. 1935. Poslední syn Antonín Karez, nar. se 20. 2. 1862. Hospodařil v č. 3 a zemřel 15. 9. 1927. Otec jejich, Antonín žil na výměnku v. č. 71 a zemřel ve věku 78 let dne 31. 7. 1908.

Antonínův syn, Josef Karez narodil se v Horní Bříze 11. 10. 1852. Jako jeho dva bratři nebyl on určen na sedlačinu. Již jako hoch se velmi dobře učil a toužil po stu­diích v Praze. Tento sen se mu však nevyplnil. Když se oženil, koupil si se svou manželkou, která pocházela z Nepomuka, chalupu v Ledcích, blízko Horní Břízy, kde měli
% str 38 @ 50
menší obchod. Byla tam však silná židovská konkurence, a tak jim šel obchod špatně. Přestěhovali se do Hor­ní Břízy a v roce 1892 dále do Plzně, Na Zavadilku. Josef Karez pracoval pak v Plzni jako dělník v Měšťanském pivovaře a v pozdějším věku byl zaměstnán u města Plzně. Zemřel tam v chorobinci (dnešní domov důchodců) 17. 7. 1938. Josef Karez měl sedm děti. Nejstarší byla Antonie, nar. 8. 4. 1882, provdala se za dílovedoucího ČSD v Plzni, Lokajíčka a zemřela 21. 12. 1925. Měla dceru Jiřinu nar. 4. 8. 1903 provd. Suchou v Chomutově, Čelakovského 24, a dalších pět dětí. Jiřina Suchá má dva syny Josefa nar. 15. 6. 1927, který bydlí také v Chomutově, Ul. Karoliny Světlé 36. Ten má dva synky Josefa, nar. 3. 1. 1953 a Jana nar. 17. 11. 1960. Další syn Jiří Suchý, nar. 27. 3. 1933 žije v Litvínově VI. č. 1617. Má synka Jiřího nar. 6. 8. 1957 a dceru Martinu, nar. 26. 9. 1964. Druhým dítětem Antonie Lokajíčkové je František. Narodil se 18. 5. 1905 a žije dnes v Plané u Mar. Lázní č. 186. Má tři syny, Františka nar. 3. 5. 1933, Zdeňka, nar. 16. 10. 1937, který je důstojníkem, bydlí ve Výškov u Tachova, má dceru Jarmilu nar. 23. 9. 1961. Třetím synem je Bohuslav Lokajíček nar. 7. 4. 1943. Třetím dí­tětem Antonie je dcera Antonie. Narodila se 17. 10. 1906 provdala se za důstojníka čsl. armády Zahradníka, a žije v Praze 6. Ul. Národní obrany č. 35. Má tři děti: její syn Zbyněk Zahradník nar. 1. 6. 1929 byl chemikem v Teplicích a v tomto povolání přivodil si předčasnou smrt 9. 5. 1960. Zanechal syna Zbyňka, nar. 18. 5. 1952, který bydlí s mat­kou v Teplicích,  ul. Prokopa Holého 3. Dcera Blanka, provdaná Vávrová, narodila se 16. 12. 1930 a bydlí s matkou v Praze 6. Má dceru Blanku, nar. 3. 6. 1952. Třetí je Antonie, provd. Lábová, nar. 15. 7. 1942 a bydlí v Praze - Bráníku, ul. Ke Krči 43. Má dceru Janu nar. 6. 5. 1963. Třetí dcera Antonie Lokajíčkové, je Marie provd. Volfová v Plzni, Lobezská č. 3. Narodila se 9. 4. 1913. (Zemřela 19. 5. 1968.) Má syna Vladi­míra nar. 12. 6. 1940 a ten má dcerku Danielu nar. v listopadu 1967. - Druhý syn Antonie je Bohuslav Lokajíček, nar. 29. 9. 1916 a bydlí v Plzni, Guldenerova 41. Jeho dcera Milada, provd. Němečková, je lékařkou. Narodila se 14. 1. 1943 a bydlí v Plzni- Koterovská 20. Má dceru Marti­nu nar. 11. 3. 1965. Druhá dcera Bohuslava Lokajíčka je Jindřiška, nar. 13. 1. 1947. Nejmladší syn Antonie je Sta­nislav Lokajíček. Narodil se 26. 11. 1923 a bydlí v Plzni, Lukavická 26. Jeho dcera Eva je nar. 5. 8. 1950.

Druhou dcerou Josefa Kareza byla Josefa. Narodila se 21. 6. 1883 v Ledcích, provdala se za zedníka Rubáše a zemřela 2. 9. 1957 v Plzni. Měla tři děti, syny. Karel Rubáš, nar. 19. 10. 1907 bydlí v Karlových Varech, nábř. A. Zápo­tockého 26. Byl čtyři roky členem MNV a velitelem požár­ního sboru z povolání. Jeho dcera Jana, zahynula malá, při náletu v Plzni. Dcera Zdeňka nar. 13. 10. 1934 provd. za architekta Kučeru, hrávala dosti dlouho v divadle Paravan. Bvdlí v Praze 1, Pařížská 19. Má dcerku Kristinu, nar. 20. 8. 1961. Jindřiška Rubášová je narozená 3. 10. 1948. Bratr Karlův, Josef Rubáš nar. se 25. 10. 1905 a zemřel svobodný 6. 6. 1949 v Broumově. Třetí syn je František Rubáš
% str 39 @ 51
nar. 10. 4. 1910. Žije dnes v Povrlech u Ústí nad Lab., Labská č. 10. V roce 1929 začal jako tanečník v plzeňském divadle u ředitele Veverky a pak působil jako herec u ředitele Drašara v divadle v Bratisla­vě a Košicích a pak v Praze, v Tylově divadle až do roku. 1948. (Dnešní divadlo v Nuslích Na Fidlovačce.) Má dceru Danielu nar. 18. 1. 1933. Ta bydlí v Plzni, sady 5. května  č. 26. Je provdaná Franková. Její neman­želský synek Zdenek Rubáš nar. 18. 7. 1952 žije v Bečově nad Teplou. Dcerka Daniela Franková nar. 27. 9. 1964.

Třetí dcerou Josefa Kareza byla Barbora, nar. 21. 2. 1885. (Zemřela 11. 6. 1949.) Provdala se za cihláře Bajera. Měla šest dětí. Nejstarší její dcera Josefa narodila se 17. 3. 1902 v Bolevci. Měla pestrý život a rodina z toho důvodu se k ní nechtěla hlásit. Zemřela 4. 8. 1962 v Praze. Dcera její, Jiřina, provd. Jermanová nar. 21. 12. 1924 bydlí v Plzni, tř. 1. máje č. 16. Je nemanželské dítě Josefy, v pozdějším věku provdané Adamové. Jiřina Jermanová má čtyři chlapce. Jiří nar. 23. 4. 1947, Ladislav nar. 7. 8. 1948, Zdeněk nar. 23. 8. 1950 a Petr nar. 2. 9. 1954. - Druhou nemanželskou dcerou Josefy Majerové byla Věra, provdaná Plevniová, nar. 25. 8. 1927. Žila v Karvinné u Ostravy a tam zemřela v mladém věku 20. 9. 1967.  - - Nejstarší syn Barbory Majerové Josef, narodil se 19. 3. 1906 a je bezdětný, bydlí v Plzni-Roudné, Lipová 10. Jeho bratr Karel nar. 24. 4. 1911 je také bez­dětný a bydlí v Plzni, ulice U trati č. 11. - Dcera Barbory, Julie provd. Voříškova, narozená 17. 5. 1914 bydlí v Plzni, Denisovo nábř. č. 8. Má dvě děti. Syn Václav nar. 15. 12. 1932 pracuje u lesní správy v Beskydech. Bydli v obci Bartošky č. 572 u Frenštátu pod Radhoštěm. Má syn­ka Václava nar. 6. 5. 1958 a dceru Zdeňku, nar. 4. 2. 1962. Dcera Olga, provd. Draská 11. 7. 1940 bydlí v Plzni, Paříž­ská 3. Má syna Bohumila nar. 6. 1. 1957 a dceru Danu nar. 6. 8. 1960. -- Třetím synem Barbory Majerové je Bohuslav. Narodil se 17. 1. 1916 a bydlí v Plzni, Prešovská č. 8. Má tři děti. Bohuslav Majer, nar. 4. 4. 1942 bydlí v Marián­ských Lázních- Úšovicích, Dobrovského 164. Má dva synky. Pavel nar. 9. 2. 1964 a Zdeněk nar. 23. 2. 1965. - Ladislav Majer nar. 21. 1. 1945 a Jindřiška nar. 4. 2. 1959. Posledním sy­nem Barbory Majerové je Ladislav. Narodil se 13. 1. 1918. Je vedoucím restaurace ve Strakonicích I.-28. Má dva syny. Pavel nar. 11. 6. 1946 a Milan nar. 5. 9. 1955.

Další dcerou Josefa Kareza byla Marie, provd. Kellerová. Narodila se 29. 1. 1888 v Horní Bříze. Její muž je dělní­kem ve Škodovce. Marie si ho vzala jako vdovce s dětmi. Z jejího manželství narodila se pak dcera Marie 16. 4. 1921. Je provdaná Jiráčková. Marie Kellerová žije v Kněžmostě č. 175 u Mnichova Hradiště. Její dcera žije v téže obci v č. 154. Marie Jiráčková má dceru Ivanu provd. Prokorátovou nar. 10. 7. 1941. Jako učitelka žije blízko v obci Branžež. Má synka Jiřího nar. 3. 3. 1964. Další dcera Marie Jiráčkové je Marta, provdaná Liebischová nar. 7. 12. 1944. Má dva syny, Václava nar. 14. 2. 1964 a Jana nar. 8. 12. 1966. Marie Jiráčková má ještě dceru Janu Lejskovou, nar. 22. 2. 1949 a synka Jiřího Jiráčka, nar. 13. 3. 1956.

% str 40 @ 52
Nejmladší dcerou Josefa Kareza je Kateřina provd. Nerudová. Narodila se 28. 8. 1890 v Horní Bříze. Její manžel je úředníkem ČSD. Žije v Písku, Na spravedlnosti č. 1481. Má jen jednoho syna, Oldřicha, nar. 15. 12. 1915. Je podplukovníkem a bydlí v Hradci Králové, Orlická Kotlina č. 1068. Má dceru Annu, nar. 28. 8. 1953.

Nejstarším synem Josefa Kareza byl Josef, nar. 16. 9. 1895 v Plzni. Byl v první světové válce a vrátil se jako ruský legionář a zůstal pak vojákem z povolání. Byl bez­dětný a jako důstojník spáchal v roce 1934 sebevraždu v Lysé nad Labem.

Posledním dítětem Josefa Kareza byl Rudolf nar. 17. 4. 1897. Byl vyučen truhlářem a v osmnácti letech šel dobrovolně na vojnu. Měl různá zaměstnání. Mimo jiné prodělal i spe­ciální školu masérskou a býval osobním masérem presiden­ta T. G. Masaryka. Po druhé světové válce přišel do Karlo­vých Varů, kde vstoupil do služeb státních lázní. Něko­lik let byl předsedou MNV. Měl dvě děti. Irena provd. Háj­ková nar. 1. 8. 1923 bydlí v Praze -Nuslích, V Horkách 13. Má dceru Janu nar. 5. 5. 1944 a Pavla nar. 5. 1. 1950. Syn Dušan Karez, narodil 1. 4. 1930. Zemřel po infarktu ve svých 32 letech 11. 11. 1962. Měl tři děti, které bydlí v Plzni, Tř. 1. máje č. 87. Jsou to: Milada nar. 21. 3. 1962, Dušana nar. 11. 8. 1955 a Kateřina nar. 28. 4. 1954. Přes politické úspěchy nebyl soukromý život Rudolfa Kareza šťastný. Na štědrý den 24. 12. 1949 zastřelil svou manželku, napsal dopis na rozloučenou a podřezal se břitvou. Jeho život byl tedy velmi pohnutý.

Druhý syn Antonína Kareza byl Václav Karez. Narodil se 8. 8. 1859 v Horní Bříze. Sedlačil pak po celý život v usedlosti č. 71 a tam také zemřel 21. ledna 1935. Měl sedm dětí, pět dcer a dva syny.

Nejstarší dcerou byla Marie nar. 7. 6. 1881 a provdala se za cihláře Majera, jehož bratra měla za manžela její sestřenice Barbora. Marie Majerová žila v Bolevci. Zemře­la tam 17. 2. 1942. Měla pět dětí. Nejstarší Josef Majer, nar. 26. 1. 1906 zůstal svobodný a bydlí v Plzni-Bolevci, Náves 34-5. Jeho sestra Albína, nar. 29. 11. 1909 měla za muže řezníka Salcmana. Je bezdětná a žije v Bolevci, jako její bratr Josef. Druhý syn je Václav Majer, nar. 2. 1. 1920. Žije v Nejdku, Osvětimská 727. Má ze dvou man­želství tři děti. Dcera Pavla provd. Peroutová nar. 28. 2. 1946 má dcerku Marcelu, nar. 21. 3. 1964, která se jmenuje po otci Kozlíková. Václav Majer narodil se 23. 7. 1947 a Jaroslav, který žije s otcem v Nejdku, je narozen 15. 9. 1948. Třetí syn Marie Majerové, Bedřich narodil se 19. 10. 1922 a dnes žije v Malé Šitboři č. 27 u Mar. Lázní. Má čtyři děti; Alena nar. 13. 5. 1950, Karel nar. 26. 10. 1952, Bedřich nar. 18. 5. 1957 a Anna nar. 14. 7. 1963. Posledním dítětem Marie Majerové je Jaroslav, nar. 29. 7. 1924, je bezdětný a žije také v Bolevci se sourozenci.

% str 41 @ 53
Druhou dcerou Václava Kareza je Anna. Narodila se 26. 8. 1885 a je provdaná Kotková v Horní Bříze č. 161. Její dcera Milada provd. Winkelhöferová je nar. 22. 5. 1909. Její dcera Věra nejdříve provdaná Poppová a nyní Žižková je narozena 2. 8. 1940. Anna Kotková zemřela 20. 5. 1968.

Nejstarším synem Václava Kareza byl Josef. Narodil se 19. 5. 1887 a hospodařil v Horní Bříze č. 45. Zemřel 3. 6. 1948. Měl dvě děti: Josef Karez nar. 5. 5. 1924 je svobodný a dcera Stanislava nar. 20. 5. 1929 je provdaná Kahounová a bydlí ve svém rodišti. Její syn Přemysl Kahoun narodil se 25. 5. 1956.

Dcera Kristina Karezová, nar. 24. 7. 1889 zemřela svo­bodná u svého otce Václava 30. 9. 1906.

Druhý syn, Václav Karez narodil se 31. 10. 1892 a padl v první světové válce na Srbsku 11. 1. 1915 .

Čtvrtou dcerou Václava Kareza je Josefa. Narodila se 9. 8. 1894 a provdala se za V. Brťka a dostala po otci grunt č. 71. Josefa Brťková je jednou z nejlepších pamětnic rodu Karezů v Horní Bříze. Zná rodové detaily nejen z dnešní doby, ale i velmi vzdálené a nesčíslné množství přesných dat, narození i úmrtí členů četné rodiny Karezů, řekne Vám zpaměti. Je to učiněný lexikon. Při obtížném hledání všech potomků Rosalie Karezové poskytla neobyčejnou pomoc tomuto historickému dílu svou pamětí. Josefa Brťková má dva syny. Václav nar. 9. 2. 1924 má dvě děti, Jarmila nar. 28. 10. 1949 a Václav nar. 20. 6. 1951. Druhý syn Josef je narozen 29. 7. 1926 a bydlí v Horní Bříze č. 364. Má synka Josefa nar. 11. 7. 1952.

Posledním dítětem Václava Kareza je Albína, provdaná Burešová. Narodila se 3. 2. 1899 a bydlí v Plzni, Chotíkovská č. 40. Má jediného syna, učitele, Oldřicha Bureše, nar. 16. 4. 1927. Ten má dva synky Petra nar. 8. 12. 1953 a Oldři­cha nar. 11. 1. 1956.

Posledním synem Antonína Kareza byl Antonín Karez. Narodil se 20. 2. 1862 v Horní Bříze a hospodařil v č. 3. Zemřel tam 15. 9. 1927 a měl jediného syna.

Jediné dítě Antonína Kareza byl Jaroslav Karez. Narodil se 24. 4. 1899 a hospodařil po svém otci na statku č. 3. Zemřel 27. 6. 1966. Měl dceru Jaroslavu, která je provdaná v Ledcích č. 53. Narodila se 14. 4. 1922. Má dvě děti Jaroslavu nar. 18. 9. 1943 a Josefa nar. 31. 12. 1948. Jaroslava poprvé provdaná Kovandová má synka Jaroslava nar. 25. 5. 1965, podruhé Neužilová synka Václava nar. 5. 3. 1968. Josef Calta má dceru Věru narozenou 3. 10. 1967.

Druhým synem Vojtěcha Kareza a Rosalie, roz.Prusíkové byl Josef Karez. Narodil se 22. 1. 1833. Hospodařil samo­statně od r. 1852 v gruntě č. 3, od něhož se dva roky předtím oddělila polovina pro jeho bratra Antonína. Josef Karez měl sedm dětí, dvě dcery a pět synů. Josef Karez zemřel ve věku 71 let dne 29. 12. 1904 ve svém domo­vě v Horní Bříze, v témže stavení kde zemřela před 53 lety jeho matka Rosalie roz. Prusíková.

% str 42 @ 54
Nejstarším dítětem Josefa Kareza z Horní Břízy č. 3 byla Josefa provd. Švábová. Narodila se 29. 2 .1852 a zemřela v Horní Bříze 19. 10. 1917. Druhý byl syn Josef Karez nar. 14. 3. 1861, který se usadil v č. 32 a zem­řel tam 6. 6. 1923. Třetí byla Barbora. Narodila se 18. 12. 1962 a zůstala svobodná. Byla kuchařkou. Zemřela 29. 1. 1910. Dalším byl Antonín, který se narodil 10. 2. 1865.  Stal se učitelem a zemřel ve Vraňanech u Měl­níka 26. 5. 1930. Syn Jakub narodil se 13. 3. 1867, byl úředníkem ČSD a zemřel tragicky mladý v roce 1905. Bratr jeho Václav, narodil se 14. 11. 1870, měl hostinec v Horní Bříze č. 17 a zemřel tam 27. 9. 1934. Posledním dítětem byl František Karez  nar. 29. 8. 1876. Byl také úředníkem na dráze jako jeho bratr. Zemřel v Jindřicho­vě Hradci 29. 1. 1946.

První dítě Josefa Kareza byla Josefa. Narodila se 29. 2. 1852 a provdala se za dělníka Švába. Žila v Horní Bříze a byla to velmi pilná a houževnatá žena. Při tehdejším špatném dopravním spojení nosila sama zboží ze své pe­kárny v nůši na zádech až do Plzně. Při jedné takové cestě byla přepadena v boleveckém lese, ubránila se, ale potom podlehla leknutí, které při přepadení utrpěla a zemřela 19. 10. 1917. Měla šest dětí. Syni až na Josefa byli pekaři. Nejstarší byla dcera Anna. Narodila se 25. 5. 1874 v Hor­ní Bříze a provdala se za hajného Čejku na polesí Hubenov u Kralovic. Měla dvě děti a při třetím porodu naro­dilo se jí mrtvé dítě a ona také dokončila svůj mladý život. Stalo se to 13. 2. 1905. Její muž padl ve světové válce. První její syn je Alois Čejka nar. 9. 3. 1900. Jako malozemědělec bydlí ve Vysoké Libyné č. 19, kde je cestářem. Má jediného syna Karla, nar. 23. 11. 1927, který bydlí v Čisté č. 184 u Rakovníka. Má tři děti: Karla nar. 19. 5. 1956, Milana na nar. 11. 7. 1957 a Květu nar. 17. 8. 1958. Druhým synem je Václav Čejka nar. 13. 9. 1903. Bydli v Modřanech u Prahy, Cholupická 320. Tam měl pěkné řeznictví. Má tři sy­ny. Nejstarší Jaroslav Čejka narodil se 7. 1. 1928, vyučený asfaltér, má dva syny Miroslava nar. 25. 11. 1953 a Ja­roslava nar. 14. 7. 1949. Druhý syn Václava Čejky je chemikem, narodil se 17. 2. 1931. Má dvě dcerky, Marii nar. 3. 9. 1959 a Markétu nar. 22. 8. 1965. Třetí, Jiří Čejka narodil se 19. 4. 1936. Je šoférem a dosud svobodný. Všichni tři bydlí s rodiči v Modřanech. Druhým, synem Josefy Švábové byl Josef. Narodil se 18. 7. 1876 a žil v Horní Bříze kde působil jako poštovní doručovatel. Zemřel tam 13. 6. 1926. Měl dvě děti. Josef Šváb narodil se 23. 2.  1921, vyučil se fotografem a měl pak různá povolání. Na začátku okupace uprchl z protektorátu přes Švýcarsko do Francie, kde se zařadil později do okupační armády americké s níž pak přišel při osvobozování domů. Nyní pracuje u Vojenských staveb. Bydlí v Praze 1, Loretánská č. 3. Z prvního rozvedeného manželství má dceru Vandu, nar. 4. 6. 1952, která žije s matkou v Příbrami č. 194. Druhým dítětem Josefa Švába je dcera Marie. Narodila se 28. 3. 1924 v Horní Bříze a má za manžela zahradníka, Šaška. Bydlí v Nové Roli u Karlových Varů, Sídliště č. 194. Má tři dcery. Jana, provdaná Pastorová nar. 6. 12. 1945. Má syna Luboše, nar. 10. 2. 1968.


% str 43 @ 55
Druhou dcerou Marie Šaškové, roz. Švábové, je Dagmar, provdaná Herzlová, naroz. 23. 11. 1947 a třetí Marie, narozená 7. 9. 1955. Třetím dítětem Josefy Švábové byl Jakub. Narodil se 19. 11. 1879 v Horní Bříze. V Plzni-Bolevci, Na Zavadilce měl hostinec a pekařství. Po brzké smrti své­ho syna Františka hostinec prodal. Jakub Šváb měl tři děti. Nejstarší je Anna nar. 29. 6. 1908 provd. Jandová a bydlí v Plzni, Krymská 15. Má jediného syna Mirosla­va nar. 21. 1. 1932. Jeho děti jsou Miroslav nar. 15. 6. 1954 a dcera Pavla nar. 15. 9. 1955. Další je Růžena, provdaná Vápeníková. Je narozena 14. 4. 1916 a je bezdětná. Bydlí v Plzni, Pallova č. 18. Syn František Šváb narodil se 1. 10. 1910 a zemřel na souchotiny ve 22 letech dne 1. 10. 1932. Jakub zemřel 20. 11. 1936.

Čtvrtým dítětem Josefy Švábové byl František. Narodil se 21. 8. 1881 v Horní Bříze a zemřel v Plzni 8. 3. 1936. Měl také hostinec jako jeho bratr Jakub. Měl tři děti, dvě dcery a syna. Nejstarší Helena provd. Pintová, naro­dila se 4. 7. 1910 a zemřela ve 39 letech v Plzni, Bolevci dne 1. 5. 1949. Měla dvě děti.František Pinta, nar. 5. 6. 1935 bydlí v Bolevci, Plasská 21. Tam také bydlí jeho sestra Eva, provdaná Krištofová narozená 10. 7. 1937. Má dceru Helenu nar. 14. 8. 1959 a syna Miloše nar. 30. 12. 1964. Druhá dcera Františka Švába je Růžena provdaná Kutzendőrferová. Narodila se 5. 6. 1912 a bydlí také v Bolevci, Plasská 21. Má dva syny. Stanislav nar. 29. 3. 1937 má dvě dcerky Marii nar. 15. 4. 1956 a Stanislavu nar. 12. 2. 1962. Druhý syn Zdeněk je narozený 23. 10. 1941 má dcer­ku Alenu nar. 7. 7. 1960. Třetím dítětem Františka Švába je syn František. Narodil se 14. 1. 1920. Je chemikem a byd­lí také v Bolevci, Plasská ul. Má čtyři děti. František, nar. 22. 1. 1942, Jaroslav nar. 27. 12. 1947, Olga nar. 11. 1. 1951 a Josef nar. 29. 5. 1953.

Poslední syn Josefy Švábové, Václav, zůstal doma. Narodil se 15. 4. 1892, byl pekařem v Horní Bříze č. 69 a zemřel tam 3. 3. 1965. Měl dva syny. Jiří Šváb nar. 7. 2. 1924 přiženil se do Obory č. 181 u Kaznějova. Je úředníkem v Plzni. Má dceru Drahoslavu nar. 8. 3. 1953.  Druhý syn Václav, je narozený 23. 8. 1932, má synka Václava nar. 14. 4. 1960 a Luďka 10. 5. 1963. Nejmladší dcerou Josefy Švábové byla Barbora. Narodila se 7. 5. 1889. Provdala se za dílenského mistra Paška. Žila dlouho v Plzni, Na Roudné 13. Zemřela tam 2. 8. 1965. Žiji tam dvě její děti. Helena provdaná Kašparová nar. 29. 5. 1919 má dceru Janu, nar. 22. 4. 1958. Dále je to syn inž. Miloslav Pašek, nar. 13. 1. 1922. Má dvě děti, Jaroslavu nar. 19. 6. 1951 a Vladimíra nar. 15. 12. 1953. Václav Šváb bydlí nyní v Plzni-Doubravka, Hrádecká 12.

Druhým dítětem Josefa Kareza, byl syn Josef. Narodil se 14. 3. 1861 v Horní Bříze č. 3.  Přiženil se pak do č. 32 v Horní Bříze, kde měl řeznictví a hostinec. Byl velmi oblíben a po dvacet let byl starostou obce. Zemřel 6. 6. 1923 ve svém domově. Měl tři děti. Prvním jeho dítětem byl syn František Karez. Narodil se 20. 8. 1895, vychodil vyšší reálku v Plzni a obchodní ško­lu. Byl pak úředníkem v Kaolince v Horní Bříze a také určitou dobu na ředitelství v Praze. Vzhledem k tomu, že se podruhé oženil se židovkou, byl v roce 1944 interno­ván v Postoloprtech. Odtud uprchl a dočkal se osvobozeni.

% str 44 @ 56
V posledních letech žil František Karez v Mýtě u Rokycan v č. 5 a zemřel tam 12. 4. 1964. Z prvního manželství měl dceru Ludmilu, nar. 9. 6. 1919. Ta je provdána za lékaře Duchka, v Kozolupech-Touškově u Plzně. Má dvě děti. Karel Duchek nar. 18. 11. 1946 a Ludmila nar. 14. 12. 1954. Dcera z druhého manželství je Asta provd. Langerová v Praze 6, Na Hubalce č. 8. Narodila se 30. 1. 1925. Ze svého prvního manželství má dceru Jiřinu Mezerovou nar. 5. 8. 1950.

Druhým dítětem Josefa Kareza byla Anna. Narodila se 16. 9. 1897. Byla učitelkou ručních prací a provdala se za finančního úředníka Pospíšila. Dlouhou dobu žila s ním na Slovensku. Tam se jí narodil syn Ladislav Pospíšil, 5. 3. 1925. Ten je technickým úředníkem v Tesle a bydlí v Praze - Hloubětíně, Poděbradská 142. Má dva syny. Ladislava nar. 22. 11. 1952 a Zdeňka nar. 18. 6. 1957. Anna Pospí­šilová zemřela krátce po své sedmdesátce 3. 10. 1967 v Praze, kde žila u syna.

Posledním dítětem Josefa Kareza, bývalého Hornobřízského starosty, byl Václav. Narodil se 31. 5. 1900 a žil se svým otcem v Horní Bříze č. 32. Zemřel 15. 9. 1963. Za­nechal po sobě jediného syna Pavla Kareza, nar. 29. 1. 1946.

Třetím dítětem Josefa Kareza byl Antonín. Narodil se v Horní Bříze č. 3 dne 10. 2. 1865. Jelikož se již v obecné škole velmi dobře učil, poslali ho rodiče na studie do Plzně. Tam studoval na gymnasiu až do kvinty a pak odešel na další studie do tehdy nově zřízeného gymnasia, v Roudnici nad Labem. Zde působil jako ředitel jeho blíz­ký příbuzný František Xaver Prusík, který se narodil ve Výrově č. 18 jako jeho babička Rosalie. Po maturitě měl Antonín Karez jíti na přání rodičů do semináře v Praze, aby se stal knězem. K tomu však neměl žádnou chut'. V je­ho dalším osudu byl mu nápomocen bratr jeho babičky, vikář Blažej Prusík, působící tehdy na Hradčanech u sv.Vita. Ten rozmluvil, jako osvícený kněz, rodičům jejich úmysl. Antonín Karez stal se pak učitelem a první jeho působiště byla škola v Zálezlicích na Mělnicku. Později učil také ve Vrbně a ve Vraňanech u Mělníka a tam také zem­řel 26. 5. 1930. Učitel Antonín Karez měl dvě dcery. Anna, naroz. 14. 6. 1902 provdala se za knihovníka v Předsednictvu vlády Julia Kettnera. Bydlí v Praze 6, Nad bárkou 12. Má dvě dcery. Anna, provdaná Němcová nar. 14. 12. 1923 v Zahrádce u Náměště nad Oslavou na Moravě. Má dvě děti. Petra nar. 16. 10. 1945 a Annu naroz. 4. 12. 1952. Druhá dcera Anny Kettnerové je inž. Jiřina Tomanová naroz. 16. 9. 1925. Žije s rodiči v Praze a je bezdětná. Druhým dítětem Anto­nína Kareza je Marie, naroz. 12. 6. 1907, provdala se za J. Němce, úředníka Hospodářského družstva v Mělníce a tam bydli v Sídlišti č 438. Má dceru Jarmilu, provd. Rezkovou, kte­rá se narodila 24. 2. 1931 a působí jako učitelka a také bydlí ve Vraňanech. Petr Němec z Náměště nad Oslavou se přiženil do velké Bíteše.

Dalším dítětem Josefa Kareza byl Jakub Karez. Narodil se 13. 3. 1867. Studoval také v Plzni a stal se později výprav­čím dráhy v Klobukách u Slaného. Byl velkým hudebníkem.

% str 45 @ 57
Když jednou orchestr, jehož byl kapelníkem, pořádal koncert v Praze, spadl s pódia, utrpěl těžké vnitřní zranění a po operaci brzy zemřel. Stalo se to v roce 1905. Nejen on, ale i celá jeho rodina měla tragický osud. Syn Miloš Karez, naroz. 10. 5. 1900 v Klobukách, zemřel jako student 19. 5. 1916 v Praze. Druhý syn František nar. 5. 4. 1903 zemřel také, když ještě studoval, 31. 3. 1917 v Praze. Sestra jejich, Julie Karezová, nar. 20. 4. 1901 zemřela jako mladá úřednice 15. 11. 1920 v Praze. Nejmladší byl Václav, který se narodil 12. 9. 1904. Byl dělníkem a žil jako samotář v Praze. Ke konci své­ho krátkého života ohluchl. Zemřel 31. 3. 1939 právě na počátku protektorátu v Praze.

Dalším synem Josefa Kareza byl Václav. Narodil se 14. 11. 1870 v Horní Bříze č. 3 a hospodařil pak v č. 17, kde měl také hostinec. Zemřel 27. 9. 1934. Měl dvě dcery. Anna provd. Kroftová narodila se 15. 11. 1896 a zemřela 5. 8. 1966. Měla 4 děti a o nich jsme vyprávěli již krátce jako o potomcích Václava Krofty, jejich otce, který byl opět potomkem Anny Prusíkové, provdané Štěpánkové v Horní Bříze.

Druhou dcerou byla Albína provdaná Kuglerová. Naro­dila se 22. 2. 1907, ale zemřela již 8.  9. 1948 v Praze. Měla dva syny. Milan Kugler narodil se 7. 10. 1928, je lékařem a bydlí v Plzni, Slovanská alej č. 8. Má syna Milana, nar. 17. 9. 1957. Druhý syn je Jaroslav Kugler, nar. 3. 12. 1931, je úředníkem Centrotexu v Praze a bydlí v Praze - Hloubětíně, Zelenečská 52. Má dceru Irenu naroz. 5. 5. 1960 a synka Jaroslava naroz. 5. 5. 1963.

Nejmladším dítětem Josefa Kareza byl syn František, naroz. 29. 8. 1876. Také on byl dán na studie. Dostal se do jižních Čech, kde působil u dráhy jako úředník ve Veselí nad Luž., v Ševětíně a naposled byl před­nostou stanice v Jarošové nad Nežárkou u Jindř. Hradce. Zemřel 29. 1. 1946 v Jindřichově Hradci a je pohřben v Třeboni. František Karez měl pět dětí. Nejstarší byl František. Narodil se 6. 12. 1900 ve Veselí nad Luž. Bvl také zaměstnán u dráhy. Dlouho působil v Pelhřimově. Tam zemřel 28. 6. 1967. František Karez patří také mezi ty, který jako výborný pamětník starých časů dopomohl ke zdaru tohoto díla. Byl dvakráte ženat. Z prvního manželství narodil se syn Roman 13. 2. 1931. Je strojním inženýrem a přednáší dnes na Vysoké škole technické v Praze. Bydlí v Praze 6, Farní 11. Je bezdětný. Bratr jeho Otomar, narozený z druhého manželství Františka Kareza narodil se 16. 10. 1933 a je technikem ve Škodovce v Plzni. Bydlí v Plzni - Slovany, ulice Mládežníků 22. Má dvě děti. Roman nar. 7. 5. 1962 a dcera Radka naroz. 9. 8. 1966.
Druhým synem Františka Kareza je Miloslav. Narodil se 15. 12. 1901, působil také u železnic celý život a Bydlí u nádraží v Třeboni. Má dva syny. Zbyněk Karez naroz. 12. 6. 1925 působí na ředitelství ČSD v Praze. Bydlí v Praze-

% str 46 @ 58

Spořilově, Jihovýchodní č. 894. Má syna Michala, naroz. 24. 1. 1958. Druhý syn je Miroslav Karez, nar. 15. 8. 1929. Je lesním úředníkem v Třeboni a tam také bydlí. Má dvě děti. Dcera  Miroslava nar. 8. 3. 1958, druhá Jarmila naroz. 11. 3. 1960.

Třetím synem Františka Kareza byl Bohuslav. Narodil se 12. 12. 1903 ve Veselí nad Luž. Byl úředníkem Úrazové Pojišťovny, byl třikráte ženat, ale šťastný život neměl. Zemřel 4. 2. 1958 v Praze. Měl dva syny. Bohuslav nar. 13. 8. 1932 jako student utekl do ciziny a jeho poslední známé bydliště bylo malé místo Glencoe ve státě Novém Jižním Walesu v Austrálii. Podle pátrání tam však dnes ji nežije a není známo, kam ho dál zavál osud. Druhý syn Bohuslavův je Milan Karez, nar. 7. 12. 1949, který bydlí s matkou v Praze, Strašnicích, Na hroudě 23.

Čtvrtým synem Františka Kareza byl Jindřich. Narodil se 3. 7. 1905 v Ševětíně. Byl učitelem a žije v důchodu v Děčíně, Máchovo nám. 2. Má dvě děti. Jaroslav je odborným učitelem a bydlí v Děčíně, Sídliště, Kamenická ul. Narodil se 17. 4. 1938 a má jednoho synka Tomáše, narozeného 4. 8. 1963. Druhé dítě Jindřichovo je dcera Jindřiška. Narodila se 4. 6. 1939, vystudovala konservatoř v Praze a je provdaná Soukupová, ale ponechala si jako operní zpěvačka své rodné jméno Karezová. Působí dnes v opeře v Českých Budějovicích. Odborný časopis, Divadelní noviny v jedné z posledních kritik o hereckých výkonech v Českých Budějovicích píše o ní toto: "Ze sólistů upoutá přirozeným hereckým projevem J. Karezová, která má i nemalé pěvecké předpoklady ve svém pěkně zbarveném, příjemně znějícím hlase." Iša Karezová, jak si říká, má synka Jana, narozeného 1. 7. 1966. Posledním dítětem Františka Kareze byla dcera Květosla­va. Narodila se 30. 4. 1910. Provdala se za inž. chemie Vladimíra Emmera a dlouho spolu žili na Slovensku. Bydlí v Praze 6, U zeměpisného ústavu č. 2. Má dvě děti. Její syn Vladimír narodil se 1. 7. 1929 a je očním lékařem v Kladně. Bydlí však v Praze 6, Na Babě, Na ostrohu 3. Má tři děti. Jarmila nar. 1. 6. 1954, Jan nar. 26. 6. 1960 a Halka Emmerová nar. 19. 12. 1962. Druhým dítětem Květoslavy je Naděžda, nar. 16. 2. 1931, je provdaná za inž. Michala, který pracuje ve Výzkumném ústavu rud. Bydlí s rodiči v Praze 6. Má dcerku Naděždu nar. 21. 12. 1953.

Třetím synem Vojtěcha Kareza a Rosalie rozené Prusíkové z Výrova, byl Jan. Narodil se v Horní Bříze 6. 8. 1838. Protože doma rozdělený grunt na polovinu dostali bratři Antonín a Josef, odešel z domova, přiženil se do Nynic na Plzeňsku, kde hospodařil. Později žil u syna ve Starém Plzenci a pak v Plzni, Na Roudné u dcery, kde zemřel 17. 5. 1918. Podle vyprávění byl to veselý člověk. Měl pět dětí. Vojtěch Kares, jak se kdysi psal jeho pradědeček, narodil se 15. 11. 1881 v Nynicích č. 12, přiženil se pak do Starého Plzence č. 15 kde hospodařil na statku a zemřel 17. 5. 1961. Druhý syn byl Josef Kares, nar. 14. 11. 1869 v Nynicích. Byl poštovním úředníkem na různých místech a
% str 47 @ 59
zemřel v Praze 30. 12. 1927. Třetí byl František Kares nar. 14. 5. 1872. Odešel do světové války jako svobodný a tam padl v r. 1915. Čtvrtý syn byl Václav Kares. Na­rodil se 14. 1. 1879 také v Nynicích, byl úředníkem Akademie výtvarných umění a Uměleckoprůmyslové školy v Praze. Zemřel v důchode v Krupce u Teplic 16. 4. 1957. Posledním dítětem Jana Karesa byla dcera Marie. Narodila se 13. 9. 1876, provdala se pak za plzeňského měšťana Beringera a zemřela v Plzni 13. 3. 1952.

Nejstarší syn Jana Karesa Vojtěch, nar. 15. 11. 1881 v Nynicích usadil se jako sedlák ve Starém Plzenci. Zemřel tam 17. 5. 1961. Měl tři děti. Nejstarší Josef Kares narodil se 10. 2. 1909 a převzal po svém otci statek a dnes je členem JZD. Bydlí ve Starém Plzenci č. 15. Má tři děti. Syn Josef Kares nar. 29. 5. 1945, Eva nar. 9. 12. 1946, a Ladislav nar. 10. 6. 1954. Druhý syn Jaroslav Kares nar. 5. 4. 1914 je dělníkem ve Škodovce. Bydlí také ve Starém Plzenci, Bezručova 547. Má dvě děti. Syna Jaroslava nar. 11. 1. 1943 a dceru Janu nar. 27. 2. 1945. Poslední dítě Vojtěcha Karesa je Anna provd. Herejková. Narodila se 11. 6. 1920. Bydlí rovněž ve Starém Plzenci, jako oba její sourozenci ve Smetanově ulici č. 114. Má syna Oldřicha, nar. 26. 4. 1947.

Druhé dítě Jana Karesa byl Josef. Narodil se 14. 11. 1869 v Nynicích, vystudoval gymnasium a stal se pak úředníkem na poště. Působil na různých místech, zvláště také dlouho v Liberci a v Praze. Zde zemřel 30. 12. 1927. Byl dvakráte ženatý. Zůstal po něm jediný syn Jaroslav, který se narodil 25. 8. 1901 v Liberci. Byl inženýrem a zaměstnán v ministerstvu dopravy, dodnes ještě vyučuje letectví na Vysoké dopravní škole v Žilině. To bylo vždycky jeho láskou. Bydlí v Praze 7, Kamenická 7 a je bezdětný. Další syn Jana Karesa Václav narodil se 14. 1. 1879 v Nynicích. Vystudoval klasické gymnasium v Plzni, ale práva na Karlově universitě v Praze nedokončil. Byl zaměstnán řadu let na Akademii výtvarných umění v Praze na Letné jako tajemník za rektora prof. Švabinského a Myslbeka. Později přešel na Umělecko-průmyslovou školu v Praze, kde byl až do pense do roku 1939.  Ke konci života usadil se v Krupce u Teplic, kde zemřel 16. 4. 1957. Měl jedinou dceru Helenu, nar. 16. 12. 1914. Ta je provdaná Slováková, je bezdětná a žije v Krupce u Teplic, Libušín č. 237. Posledním dítětem Jana Karesa byla dcera Marie. Narodila se 13. 9. 1876 v Nynicích a později se provdala za podnikavého měšťana plzeňského Beringera. Marie zemřela 13. 3. 1952 v Plzni Na Roudné. U ní také zemřel v r. 1918 její otec. Měla jediného syna Bohuslava Beringera. Narodil se 30. 3. 1896 v Plzni a tam zemřel 30. 3. 1954. Ten má dva syny. Miroslav Beringer je narozen 22. 5. 1919 a bydlí v Plzni, Pod záhorskem č. 9. Má dva syny, Jiřího nar. 7. 12. 1945 a Jaromíra nar. 15. 7. 1953. Druhý syn Marie Beringerové je Jaroslav nar. 26  9. 1928. Bydlí ve stejném domě jako jeho bratr. Má rovněž dva syny, Miloše Beringera nar. 19. 5. 1956 a Jiřího nar. 14. 7. 1958.

% str 48 @ 60
Posledním dítětem Vojtěcha Kareza a Rosalie, roz. Prusíkové z Výrova byla dcera Marie. Narodila se ještě za roboty 12. 2. 1843. Byla jedinou dcerou, která dospěla. Sestra její Barbora zemřela ještě malá. V roce 1865 provdala se za sedláka  Vojtěcha Maška do nedaleké Třemošné č. 17. Měla velmi četné potomstvo, devět jejích dětí dospělo. Nejstarší byl syn Josef nar. 11. 2. 1866, převzal později rodný statek, ale zůstal bezdětný. Aby nepřišla usedlost do cizích rukou, jak se říkalo, předal ještě za života grunt dceři, po své brzy zemřelé sestře Marii Kimmlové, Anně, která byla později provda­ná Andrlíková. Josef Mašek zemřel v Třemošné 8. 5. 1933. Druhým dítětem byl František, narodil se 20. 12. 1869, byl pekařem v Plzni, Bolevci a zemřel 2. 8. 1946. Třetí byl Václav Mašek, který se narodil 6. 5. 1873, zůstal v Třemošné a zemřel 14. 1. 1933. Čtvrtým synem byl Antonín Mašek nar. 26. 3. 1875, žil v Plzni a zemřel 30. 3. 1941. Pátým dítětem byla dcera Marie, nar. 18. 9. 1879 v Třemošné, provdala se za zemědělce Kimmela v Ondřejově u Plas. Zemřela tam mladá 13. 4. 1916. Další byl Jakub Mašek nar. 26. 6. 1881 v Třemošné, usazený pak jako pekař v Plzni, Škvrňanech, kde zemřel 22. 3. 1958. Pak byla dcera Anna. Narodila se 19. 9. 1883 v Třemošné, byla provdaná Ehegartnerová v Plzni, Na Roudné. Dožila se nejdelšího věku ze svých sourozenců, zemřela 27. 3. 1963 v nemocnici v Rokyca­nech. Třetí dcerou byla Josefa Mašková. Narodila se 31. 3. 1889, byla provdaná Harmáčková v Plzni, kde zemřela 23. 7. 1952. Poslední byl Martin Mašek nar. 3. 2. 1891 v Tře­mošné. Žil pak v Horní Bříze, kde jeho manželství by­lo bezdětné. Zemřel tam 27. 9. 1957. Přesto, že byla Marie Mašková matkou tolika dětí, zemřela 84 letá, 16. června 1924.

O prvním synu Vojtěcha a Marie Maškových z Třemošné, Josefovi, jsme se již zmínili. Neměl potomků.

Jeho bratr František Mašek narodil se 20. 12. 1869. Měl pekařský závod v Plzni, Bolevci a byl to typ starého českého, poctivého, bodrého i bystrého venkovana. Zvlášt­ní jeho zálibou byla hra na dudy. Naučil se tomu od své­ho strýčka v Třemošné, který jako mladý hoch při hře "na špačky" přišel o zrak. František Mašek se ve hraní na dudy tak zdokonalil, že se stal pozorností hudebních odborníků, jimž předzpíval a zahrál na dudy mnohé staré a neznámé písničky. Zachyceny jsou na gramofonových deskách. František Mašek měl pět dětí. Zemřel 2. 8. 1946 v Plzni. Nejstarší byla dcera Eliška. Narodila se 20. 12. 1903, provdána byla Vavříková v Plzni, kde zemřela v červenci 1954. Její syn inž. František Vavřík nar. 20. 7. 1922 bydlí v Plzni, Mánesova 32. Má dva syny Romana, nar. 27. 8. 1952 a Jiřího nar. 23. 3. 1954. Sestra její Františka narodila se 28. 6. 1902 a bydlí v Plzni, Bolevci, Frunzeho 3. Má dceru Emilii provd. Seberovou, nar. 3. 5. 1924. Ta má syna Karla, nar. 24. 9. 1949. Syn Vojtěch Mašek nar. 21. 3. 1904 zemřel svobodný již 28. 3. 1925. Dcera Emilie nar. 4. 1. 1906 byla provdaná Bernášková a zemřela 28. 3. 1964. Měla jediného syna Václava Bernáška, nar. 5. 8. 1934. Bydlí v Plzni, Bolevci, Frunzeho č. 3. Má dceru Jindřišku nar. 22. 12. 1958.

% str 49 @ 61
Posledním dítětem Františka Maška je Cecilie. Narodila se 15. 10. 1911 a je provdaná za obchodníka Kunzmana. Pěkný obchod není již jejich majetkem. Bydlí v Plzni-Bolevci, Lidická 42. Její dcera Jindřiška nar. 20. 5. 1934 je provdaná za lékaře Máchovského, Má synka Josefa, nar. 2. 2. 1955.

Dalším dítětem Vojtěcha a Marie Maškových v Třemošné byl Václav. Narodil se 6. 5.  1873. Když se oženil dali mu rodiče část polí a vystavěli nové obytné stavení na témže dvoře. Václav Mašek měl 4 děti. Zemřel 14. 1. 1933 v Třemošné. První jeho dítě, dcera Marie narodila se 29. 1. 1902. Je provdaná Roubalová a žije v České Bříze  č. 69. Má jednu dceru Vlastu nar. 21. 7. 1925, která je provdaná Pešková v Plzni - Letné, Revoluční 57. Ta má dva syny Jiřího nar. 20. 9. 1949 a Zdeňka Peška nar. 6. 5. 1954. Druhou dcerou je Františka, provd. Denková a bydlí v Třemošné č. 62. Narodila se 9. 12. 1906. Má dvě děti. Dcera Zdeňka provdaná Jílková je narozena 29. 8. 1935 a bydlí v Plzni - Zátiší, Línská 5. Má dva syny: Karla nar. 21. 8. 1957 a Zdeňka nar. 21. 6. 1964. Václav Denk nar. 13. 3. 1942 žije v Třemošné.  Syn Václav Mašek nar. 15. 9. 1910 žije v Třemošné č. 129. Je zedníkem. Když se pátralo p0 osudech neobyčejně rozvětvené rodiny Ma­rie Maškové, roz. Karezové, která je členem našeho rodu Prusíků, byl Václav Mašek prvním, kdo uvedl na stopu a tím vytvořil předpoklady k nalezení všech potomků Ma­rie Maškové. Václav Mašek má jednu dceru Miluši provd. Dobrou v Třemošné č. 733. Je narozena 11. 8. 1934 a má synka Václava nar. 23. 9. 1955 a Irenu nar. 8. 2. 1959. Posledním dítětem Václava Maška je Zdeňka. Narodila se 22. 7. 1913 a je provdaná Pašková v Trnové č. 10. Její dcera Miluše je provdaná Zelenková, bydlí v Trnové. Narodila se 9. 7. 1940 a má dceru Pavlu nar. 3. 4. 1965. Jiří Pašek, nar. 6. 12. 1942 přiženil se do Kačerova č. 7. Má synka Ji­řího nar. 29. 8. 1966. Dále je Anna Pašková, nar. 13. 7. 1948 a Zdeňka nar. 20. 6. 1953.

Antonín Mašek, další dítě Vojtěcha a Marie Maškové z Třemošné, narodil se tam 26. 3. 1875. Žil v Plzni a tam zemřel 30. 3. 1941. Jeho jediný syn Antonín narodil se 21. 10. 1901, byl v Plzni úředníkem a zemřel v domově dů­chodců v Liblíně, bez potomků 30. 7. 1961.

První dcerou Marie Maškové roz. Karezové byla Marie. Narodila se 18. 9. 1879 v Třemošné. Provdala se brzy za zemědělce Kimmela do Ondřejova blízko Plas. Tato rodina měla však špatný osud. Její muž padl na začátku první světové války v Srbsku a ona zemřela již 13. 4. 1916 také velmi mladá. Měla šest dětí a všechny tedy brzy osiřely. Starali se pak o ně babička s dědečkem z Třemošné. Nejstarší byla Anežka. Narodila se 28. 2. 1904 a zůstala jako provdaná Hauznerová, ve svém domově v Ondřejově č. 4. Má dceru Boženu nar. 27. 1. 1923. Ta je provdaná Sinkulová a bydlí v blízkém Křečově č. 13. Její dcera Pavla, provd. Sachrová je narozena 8. 1. 1946. Má dcerku Petrušku nar. 13. 1. 1965. Další dcera Boženy je Valentina nar. 20. 9. 1950.

% str 50 @ 62
Bohuslav Hauzner nar. 6. 1. 1927 a žijící v Ondřejově č. 4  je bezdětný. Další byla Marie. Narodila se 18. 3. 1905 je pro vdaná za zaměstnance ČSD Kubelíka. Bydlí v Mladoticích, na zastávce č. 99. dceru Věru nar. 14. 12. 1932. Je provdaná Hofraitrová a bydlí v Plzni - Slovanech, U Lomu č. 13. Má dceru Evu nar. 8. 9. 1953. Ludmila narodila se 8. 9. 1906 v Ondřejově a jako provdaná Kriegrová bydlí v Borku u Žlutic č. 6. Má tři děti. Ladislav Kriegr nar. 6. 6. 1927 bydlí také v Borku. Má dva syny Miloslava nar. 2. 5. 1961 a Jiřího nar. 16. 8. 1962. Václav Kriegr na­rodil se 23. 11. 1932. Je usazen také v Borku. Má tři sy­ny. Václava nar. 1. 10. 1957, Ladislava nar. 26. 6. 1961 a Petra nar. 23. 6. 1964. Dcera Marie je provdaná Soukupová, narodila se 26. 10. 1928 a žije ve Stodě u Plzně č. 595. Má dceru Janu, nar. 6. 7. 1956. Další je Anna. Narodila se 25. 3. 1908 v Ondřejově, ale ještě za první světové války vzali si ji jako sirotka do Třemošné dědeček s babičkou. Když její strýc, Josef Mašek zůstal bezdětný, předal ji pak, když se vdala za sedláka Andrlíka, hospodářství. Na tomto gruntě v Třemošné č. 17 žila  Anna Andrlíková dosud. Má jednu dceru Andělu, nar. 29. 7. 1929. Je provdaná Hájková a bydlí v Třemošné, v ulici Ke staré cihelně č. 285. Má dvě děti, Pavla nar. 5. 7. 1949 a Alenu nar. 7. 3. 1956. Dne 24. 4. 1909 narodil se v Ondřejově první syn Jaroslav Kimmel. Bydlí v Plzni - Bolevci, Pod Mihulkou č. 273. Jeho dcera Jaroslava, provd. Bělíková je nar. 17. 9. 1933 a má dceru O1gu nar. 17. 3. 1953. Syn Václav Kimmel, nar. 8. 7. 1942 má synka Jaroslava nar. 11. 6. 1965. Druhý syn Marie Kimmelové, Václav byl zaměstnán u ČSD. Narodil se  29. 11. 1911 v Ondřejově, byl ženatý, ale zůstal bezdětný a zemřel poměrně mladý 10. 6. 1959 v Praze, kde je pochován. Jaroslava Bělíková bydlí v Praze 6, Leninova 676.

Dalším synem Vojtěcha a Marie Maškových z Třemošné, byl Jakub. Narodil se 26. 6. 1881 a byl pekařem. Byl usazen v Plzni - Škvrňanech, Hornická 66, kde bydlí dosud jeho syn František. Jakub měl čtyři děti. Zemřel v Plzni 22. 3. 1958. První byla Anna nar. 26. 7. 1906, je provdaná Vildmanová a bydlí v Plzni, tř. 1. máje 133. Má dvě dcery. Irena je narozená 20. 4. 1933 je provdaná Bernatová a bydlí v Plzni - Slovanech, Slovanská 81. Má synka Ivo, nar. 5. 1. 1960. Druhá dcera Eva je provdaná Votípková, je narozená 11. 4. 1943. Má syna Karla nar. 5. 11. 1965. Syn František narodil se 6. 2. 1908. Vyučil se pekařství u svého otce Jakuba, bydlí v Plzni - Škvrňanech, Hornická 66. Má dvě dcery. Jiřina provd. Němcová je narozená 21. 4. 1930 a bydlí v Plzni, Koperníkova 32. Má syna Jiřího nar. 18. 9. 1951 a dceru Václavu nar. 13. 6. 1954. Druhá dcera Františka Maška je Zdeňka nar. 16. 11. 1928. Jako provd. Reiserová bydlí v Č. Budějovicích, Plzeňská. 19. Má dva syny, Aloise nar. 28. 8. 1952 a Petra nar. 5. 11. 1964. Druhý syn Jakubův, Josef Mašek nar. 22. 2. 1912 zemřel jako bezdětný v Plzni 9. 6. 1958. Čtvrtý je Ladislav Mašek nar. 10. 5. 1914 a je úředníkem ČSD. Bydlí v Praze - Vinohradech, Bělehradská č. 22.

% str 51 @ 63
Marie Mašková měla dále dceru Annu. Narodila se 19. 9. 1883 v Třemošné a byla provdaná Ehegartnerová. Dosáhla nejdelšího věku ze svých sourozenců. Téměř osmdesáti­letá zemřela v rokycanské nemocnici 27. 3. 1963 . Měla dva syny, kteří si pozměnili jméno na Zahradník. Josef Zahradník nar. 17. 2. 1906 vyučil se truhlářem. Bydlí v Plzni - Bolevci, Lidická ul. 58. Zemřel 30. 12. 1965. Měl syna Miloše nar. 17. 12. 1940 a dceru Marii, která je provdaná Tvrdíková. Je narozena 23. 3. 1946 a bydlí v Plzni, jako její matka vdova. Má dcerku Lenku nar. 30. 6. 1966. Druhý syn, Václav Zahradník je narozen 24. 6. 1914 a je požárníkem u ČSD. Bydlí v Plzni, Pittnerova č. 11. Má syna Václava naroz. 24. 1. 1940, který bydlí v Plzni. Nepomucká 2. Ten má dvě dcerky; Irenu naroz. 29. 3. 1960 a Danu naroz. 9. 4. 1962. Druhá dcera je Božena Zahradníková, naroz. 18. 7. 1945. Bydlí s rodiči.

Poslední dcera Marie Maškové byla Josefa. Narodila se 31. 3. 1889 v Třemošné a provdala se za malíře pokojů, Harmáčka. Zemřela v Plzni 23. 7. 1952. Měla dvě děti. Dcera Marie je naroz. 25. 12. 1905 a žije jako provdaná Kovaříková, ale bezdětná, v Plzni, Tylova 17. Její bratr Rudolf Harmáček narodil se 20. 2. 1910. Má dvě dě­ti. Jeho syn Rudolf nar. 27. 11. 1942 a dcera Vlasta provd. Biberlová je narozena 17. 2. 1937. Bydlí v Plzni- Škvrňanech, Emingerova 18. Má dceru Martu nar. 3. 3. 1962 a syna Vlastimila nar. 10. 5. 1964.
Posledním synem Maškových z Třemošné č. 17 byl Martin. Narodil se 3. 2. 1891, oženil se a odstěhoval do Horní Břízy. Dětí však neměl. Zemřel tam 27. 9. 1957.

Tím jsme skončili krátké líčení o životě a osudech potomků Rosalie Karezové z Horní Břízy,  která se narodila v roce 1807 ve Výrově u Kralovic. Její otec, Vojtěch Prusík byl tam rychtářem. Jako její starší sestra Anna, provdaná Štěpánková, také v Horní Bříze, měla i ona, četné potomstvo. Na začátku roku 1968, kdy jsme toto vyprávění skončili, je opravdu počet potomků značný. Jde o 378 lidí z nichž 70 již zemřelo.

% str 51+1 @ 64
% TODO foto

\end{document}
